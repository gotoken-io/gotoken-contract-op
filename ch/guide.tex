\section{操作帮助}
本节对于常见的操作给出通用的合约操作方法,本节描述的方法仅针对直接操作合约,以EtherScan为例。
\subsection{查看多签时使用的\texttt{id}}
在进行多签操作时,通常需要一个\texttt{id},此\texttt{id}用于标识试图通过达成的共识的编号,该编号有以下特点:
\begin{enumerate}
\item 该\texttt{id}必须是从1开始连续递增的,不能跳过编号;
\item 一个多签人仅能对同一个\texttt{id}进行一次表决;
\item 该\texttt{id}的编号对于不同的函数是独立的,互相不影响;
\item 该\texttt{id}的编号是由多签合约提供的,因此多签合约更换时,相应的\texttt{id}需要重新开始编号。
\end{enumerate}
综合上述特征,不难发现,在多签操作前,取得正确的\texttt{id}是很重要的。

对于任一非多签合约,应该都可以在EtherScan中找到其对应的多签合约(合约中的\texttt{multisig\_contract}地址),
在多签合约中,调用(read contract)\texttt{get\_unused\_invoke\_id},填入要进行多签的函数名,就可以得到相应的\texttt{id}。

\subsection{如何进行铸币}
铸币的前提是铸币人需要在铸币白名单中。

铸币操作分为两步:
\begin{enumerate}
\item 调用USDT合约的\texttt{approve}方法,需要注意的是,其中的spender必须为FundAndDistribute合约的地址,数量必须大于需要铸币的数量,例如\\
\ccontract{0xdAC17F958D2ee523a2206206994597C13D831ec7}{approve}{0xAA9a51f48834924B2F79639Af74FefE9BFF74529, \\10000000000} \\
其中,调用的地址为USDT合约的地址。

\item 调用FundAndDistribute合约的\texttt{fund}方法,需要注意的是,由于该方法需要进行多次转账,因此需要适当调高gas费用。
\end{enumerate}